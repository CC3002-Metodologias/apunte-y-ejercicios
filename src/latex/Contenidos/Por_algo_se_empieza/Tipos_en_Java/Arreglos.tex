\subsection{Arreglos}
  Además de los tipos primitivos tendremos otro tipo (o más bien una estructura) que debemos 
  conocer.
  Los \textit{arreglos} son un conjunto ordenado de valores.

  Los arreglos son objetos que comparten la propiedad de los tipos primitivos de ocupar una posición
  fija y constante en la memoria.
  Con los tipos primitivos es fácil entender como un conjunto fijo de éstos ocupará siempre la misma
  cantidad de memoria pero cuando tenemos arreglos de objetos no es tan obvio y para entenderlo se
  necesita entender el concepto de punteros y referencias, pero esto se escapa del alcance de este 
  libro.\autocite{java-nutshell-syntax-arrays}

  Dos conceptos que suelen confundirse son el de arreglo y lista.
  En \textit{Python} no se hace diferencia entre lo que es un arreglo y lo que es una lista (en 
  estricto rigor, \textit{Python} no tiene arreglos), pero en otros lenguajes como \textit{Java} y 
  \textit{C++} son fundamentalmente distintos y es importante saber cuándo usar uno u otro.
  Más adelante volveremos a retomar el concepto de lo que es una lista, pero ahora veamos cómo usar
  los arreglos.
  Existen contextos en los que se refiere a los arreglos como listas, pero eso es incorrecto; más
  adelante veremos más en detalle lo que es una lista y por qué es importante hacer la diferencia
  con los arreglos.