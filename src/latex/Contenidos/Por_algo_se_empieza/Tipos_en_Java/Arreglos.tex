\subsection{Arreglos}
  Además de los tipos primitivos tendremos otro tipo (o más bien una estructura) que debemos 
  conocer.
  Los \textit{arreglos} son un conjunto ordenado de valores.

  Los arreglos son objetos que comparten la propiedad de los tipos primitivos de ocupar una posición
  fija y constante en la memoria.
  Con los tipos primitivos es fácil entender como un conjunto fijo de éstos ocupará siempre la misma
  cantidad de memoria pero cuando tenemos arreglos de objetos no es tan obvio y para entenderlo se
  necesita entender el concepto de punteros y referencias, pero esto se escapa del alcance de este 
  libro.\autocite{java-nutshell-syntax-arrays}

  Dos conceptos que suelen confundirse son el de arreglo y lista.
  En \textit{Python} no se hace diferencia entre lo que es un arreglo y lo que es una lista (en 
  estricto rigor, \textit{Python} no tiene arreglos), pero en otros lenguajes como \textit{Java} y 
  \textit{C++} son fundamentalmente distintos y es importante saber cuándo usar uno u otro.
  Más adelante volveremos a retomar el concepto de lo que es una lista, pero ahora veamos cómo usar
  los arreglos.
  Existen contextos en los que se refiere a los arreglos como listas, pero eso es incorrecto; más
  adelante veremos más en detalle lo que es una lista y por qué es importante hacer la diferencia
  con los arreglos.

  La sintaxis para crear un arreglo es \mintinline{java}{type[] name = new type[size]}.
  Un ejemplo podría ser:
  \begin{minted}{java}
    int[] integers = new int[3];
  \end{minted}

  Lo que hace el código anterior es crear un arrreglo de 3 números enteros, o más bien, reservar la
  memoria para 3 números.
  Luego, podríamos \enquote{llenar} el arreglo de la siguiente forma:
  \begin{minted}{java}
    integers[0] = 11;
    integers[1] = 420;
    integers[2] = 8000;
  \end{minted}

  Pero se imaginarán que iniciar así uno a uno los valores terminaría en muchas líneas de código.
  Otra forma de hacer esto sería introduciendo un ciclo, pero para arreglos relativamente pequeños
  \textit{Java} tiene una sintaxis más compacta:
  \begin{minted}{java}
    int[] integers = new int[]{11, 420, 8000};
  \end{minted}

  Noten que en esta segunda forma no especificamos el tamaño del arreglo ya que el compilador lo 
  puede saber fácilmente.

  \begin{important}
    Una última cosa importante a tener en cuenta es que en \textit{Java} \textbf{los arreglos son 
    homogeneos}, esto significa que no podemos crear arreglos que tengan elementos de tipos distintos.
  \end{important}