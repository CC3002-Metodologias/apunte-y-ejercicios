% !TEX root = D:\Users\Ignacio\Documentos\Escuela\CC3002 - Metodologías de Diseño y Programación\apunte-y-ejercicios\src\latex\Apunte.tex
\subsection{Constructores y herencia}
  Ya hemos mencionado anteriormente cómo la herencia es una de las propiedades más poderosas de la
  programación orientada a objetos, entonces una pregunta natural sería cómo funcionan los 
  constructores cuando hay herencia.

  Tomemos como punto de partida el ejemplo de vectores que vimos en el capítulo 
  \ref{ch:python-java}.

  Primero modifiquemos un poco la clase \texttt{VectorND} para integrar los modificadores de 
  privacidad que vimos en la sección anterior:

  \begin{minted}{java}
    public class VectorND {
      protected double[] tail;

      public VectorND(double[] tail) {
        this.tail = tail;
      }

      public VectorND add(VectorND otherVector) {
        // Igual que antes
      }

      public void print() {
        // Igual que antes
      }
    }
  \end{minted}
  
  TODO:
  \begin{itemize}
    \item Llamada implícita a super
    \item Herencia de constructores
  \end{itemize}
