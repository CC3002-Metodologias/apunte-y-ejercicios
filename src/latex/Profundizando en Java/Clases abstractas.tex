% !TEX root = C:\Users\smfla\Documents\Escuela\CC3002 - Metodologías de Diseño y Programación\apunte-y-ejercicios\src\latex\Apunte.tex
\section{Clases abstractas}
  En las secciones anteriores creamos una clase \texttt{Bakemon} de la que extendían tipos 
  específicos de \textit{Bakémon}, esto no está mal.
  ¿Pero tiene sentido crear instancias de la clase \texttt{Bakemon}?
  O dicho de otra forma: ¿Existe algún contexto en el que necesitaríamos un \textit{Bakémon} sin 
  tipo?
  La verdad es poco probable.

  Para entender este problema alejémonos un poco del ejemplo anterior y vamos a algo más simple, si
  tenemos un tenedor, una cuchara y un chuchillo, entonces podemos agruparlos diciendo que todos son 
  utensilios.
  ¿Pero qué es un utensilio?
  ¿Existe algún utensilio que no sea un cuchillo, una cuchara, un tenedor u otro?
  Replanteemos esta última pregunta: ¿Existe algún elemento que sea un utensilio solamente?
  La respuesta es no, no existe una representación concreta (o material) de un utensilio, diremos
  entonces que es un \textbf{concepto abstracto}.

  Retomando el ejemplo original, un \textit{Bakémon} sería un concepto abstracto, ya que engloba al
  conjunto de \textit{Bakémon}, pero no tiene una forma concreta ya que todos los \textit{Bakémon}
  tienen un tipo.

  En \textit{Java} para representar conceptos abstractos existen las \textbf{clases abstractas}.
  Una clase abstracta es una clase que \textbf{no puede ser instanciada}, i.e. no puede recibir el
  mensaje \mintinline{java}{new}.
  Aún así, las clases abstractas pueden tener constructores porque, a pesar de no poder recibir el
  mensaje \mintinline{java}{new}, sí pueden recibir el mensaje \mintinline{java}{super(...)}.

  Para definir una clase abstracta se utiliza la \textit{keyword} \mintinline{java}{abstract}, con
  lo que podemos reescribir nuestra clase \texttt{Bakemon}\footnote{Es una buena práctica que el 
  nombre de las clases abstractas comience con \texttt{Abstract}.} como:

  \begin{minted}{java}
    public abstract class AbstractBakemon {
      private String name;
      private String type;
      private int hitPoints;

      protected AbstractBakemon(String name, String type, int hitPoints) {
        this.name = name;
        this.type = type;
        this.hitPoints = hitPoints;
      }

      public String getName() {
        return name;
      }

      public String getType() {
        return type;
      }

      public void increaseHitPointsBy(final int amount) {
        hitPoints += amount;
      }

      public int getHitPoints() {
        return hitPoints;
      }

      public void setHitPoints(int hitPoints) {
        hitPoints = hitPoints;
      }
    }
  \end{minted}

  Noten que el único cambio que hicimos fue agregar la \textit{keyword} \mintinline{java}{abstract}
  a la definición de la clase.
  Con esto podemos reutilizar las funciones de esta clase y evitar la duplicación de código, pero,
  como veremos ahora y en lo que sigue del curso, es crucial siempre cuestionar nuestro diseño.

  En el capítulo \ref{ch:oop} dijimos que el fin de utilizar herencia \textbf{no es} evitar la 
  duplicación de código, sin embargo esa es la única función que está cumpliendo nuestra clase 
  abstracta ahora.
  Para que tenga sentido utilizar clases abstractas entonces debe existir al menos una funcionalidad
  que no esté implementada en la clase abstracta y que se le delegue esa funcionalidad a sus hijos,
  esto es lo que se conoce como \textit{métodos abstractos}.

  Un \textbf{método abstracto} es un método que se define en una clase abstracta pero se implementa
  en las clases que heredan de ésta.
  En la práctica, un método abstracto se define como la firma del método seguido de un \texttt{;}.

  Ahora, del ejemplo anterior hay un método que debiera ser abstracto.
  Si se fijan, el método \texttt{getType()} será distinto para todas las clases que la extiendan y,
  por lo tanto, debiera ser abstracto, así podemos redefinir la clase como:
  
  \begin{minted}{java}
    public abstract class AbstractBakemon {
      private String name;
      private int hitPoints;
      private int hitPoints;

      protected AbstractBakemon(String name, int hitPoints) {
        this.name = name;
        this.hitPoints = hitPoints;
      }

      public String getName() {
        return name;
      }

      public abstract String getType();

      public void increaseHitPointsBy(int amount) {
        hitPoints += amount;
      }

      public int getHitPoints() {
        return hitPoints;
      }

      public void setHitPoints(int hitPoints) {
        hitPoints = hitPoints;
      }
    }
  \end{minted}

  Y ahora modificamos sus clases hijas de la siguiente forma.

  \begin{minted}{java}
    public class WaterBakemon extends AbstractBakemon {
      public WaterBakemon(String name, int hitPoints) {
        super(name, hitPoints);
      }

      @Override
      public String getType() {
        return "WATER";
      }
    }
  \end{minted}

  Noten que eliminamos el parámetro \texttt{type}, una mala práctica común es tener métodos y 
  variables innecesarias en nuestro código, esto puede llevar a un mal entendimiento y uso del 
  programa así que es mejor eliminarlas.

  Un último detalle importante sobre las clases abstractas es que todos los métodos abstractos 
  \textbf{deben} implementarse en las clases que hereden de ésta (ya sea directa o indirectamente).
  Si no se hace esto, el programa no va a compilar.

  \begin{exercise}
    ¿Tiene sentido tener un constructor público en una clase abstracta?
    ¿Y uno privado?
  \end{exercise}
%