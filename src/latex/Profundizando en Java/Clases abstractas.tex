\section{Clases abstractas}
  En las secciones anteriores creamos una clase \texttt{Pokemon} de la que extendían tipos 
  específicos de \textit{Pokémon}, esto no está mal.
  ¿Pero tiene sentido crear instancias de la clase \texttt{Pokemon}?
  O dicho de otra forma: ¿Existe algún contexto en el que necesitaríamos un \textit{Pokémon} sin 
  tipo?
  La verdad es poco probable.

  Para entender este problema alejémonos un poco del ejemplo anterior y vamos a algo más simple, si
  tenemos un tenedor, una cuchara y un chuchillo, entonces podemos agruparlos diciendo que todos son 
  utensilios.
  ¿Pero qué es un utensilio?
  ¿Existe algún utensilio que no sea un cuchillo, una cuchara, un tenedor u otro?
  Replanteemos esta última pregunta: ¿Existe algún elemento que sea un utensilio solamente?
  La respuesta es no, no existe una representación concreta (o material) de un utensilio, diremos
  entonces que es un \textbf{concepto abstracto}.

  Retomando el ejemplo original
%