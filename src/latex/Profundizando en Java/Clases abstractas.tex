% !TEX root = C:\Users\smfla\Documents\Escuela\CC3002 - Metodologías de Diseño y Programación\apunte-y-ejercicios\src\latex\Apunte.tex
\section{Clases abstractas}
  En las secciones anteriores creamos una clase \texttt{Pokemon} de la que extendían tipos 
  específicos de \textit{Pokémon}, esto no está mal.
  ¿Pero tiene sentido crear instancias de la clase \texttt{Pokemon}?
  O dicho de otra forma: ¿Existe algún contexto en el que necesitaríamos un \textit{Pokémon} sin 
  tipo?
  La verdad es poco probable.

  Para entender este problema alejémonos un poco del ejemplo anterior y vamos a algo más simple, si
  tenemos un tenedor, una cuchara y un chuchillo, entonces podemos agruparlos diciendo que todos son 
  utensilios.
  ¿Pero qué es un utensilio?
  ¿Existe algún utensilio que no sea un cuchillo, una cuchara, un tenedor u otro?
  Replanteemos esta última pregunta: ¿Existe algún elemento que sea un utensilio solamente?
  La respuesta es no, no existe una representación concreta (o material) de un utensilio, diremos
  entonces que es un \textbf{concepto abstracto}.

  Retomando el ejemplo original, un \textit{Pokémon} sería un concepto abstracto, ya que engloba al
  conjunto de \textit{Pokémon}, pero no tiene una forma concreta ya que todos los \textit{Pokémon}
  tienen un tipo.

  En \textit{Java} para representar conceptos abstractos existen las \textbf{clases abstractas}.
  Una clase abstracta es una clase que \textbf{no puede ser instanciada}, i.e. no puede recibir el
  mensaje \mintinline{java}{new}.
  Aún así, las clases abstractas pueden tener constructores porque, a pesar de no poder recibir el
  mensaje \mintinline{java}{new}, sí pueden recibir el mensaje \mintinline{java}{super(...)}.

  Para definir una clase abstracta se utiliza la \textit{keyword} \mintinline{java}{abstract}, con 
  lo que podemos reescribir nuestra clase \texttt{Pokemon} como:

  \begin{minted}{java}
    public abstract class AbstractPokemon implements IPokemon {
      private final String name;
      private final String type;

      protected AbstractPokemon(String name, String type) {
        this.name = name;
        this.type = type;
      }

      @Override
      public String getName() {
        return name;
      }

      @Override
      public String getType() {
        return type;
      }
    }
  \end{minted}
%