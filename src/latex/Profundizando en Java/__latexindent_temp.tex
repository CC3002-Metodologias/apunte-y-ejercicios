% !TEX root = D:\Users\Ignacio\Documentos\Escuela\CC3002 - Metodologías de Diseño y Programación\apunte-y-ejercicios\src\latex\Apunte.tex
\section{Convenciones}
  Antes de explicar cómo funciona \textit{Java}, veamos las convenciones de cómo escribir
  código.
  Estas convenciones no es necesario seguirlas, pero es \textbf{altamente recomendado} ya
  que hace más fácil estandarizar la forma en que se escribe un programa cuando se trabaja
  en equipos.

  Las convenciones que ilustraremos aquí son las que utilizan los programadores de 
  \href{https://google.github.io/styleguide/javaguide.html}{\textit{Google}}, ya que es 
  uno de los estándares más usados y completos.

  \subsubsection{Archivos}
    
  \rule{\linewidth}{1pt}

    \paragraph{Nombre de archivos}
      El nombre de los archivos debe ser el nombre de la clase principal del archivo 
      (cada archivo debe tener \textbf{a lo más} una clase principal) con la terminación
      \texttt{.java}.

      Por ejemplo, la clase \texttt{VectorND} que definimos anteriormente debe estar en un 
      archivo llamado \texttt{VectorND.java}.
    %
    \paragraph{\textit{Encoding}}
      El \textit{encoding} de los archivos debe ser \textit{UTF-8}.
    %
    \paragraph{Espacios en blanco}
      Aparte del salto de línea (que dependerá del S.O.), solamente debe usarse el espacio
      estándar \textit{ASCII} (\texttt{0x20}).
      Esto implica que no deben utilizarse \textit{tabs} para indentar el código.
    %
  %
  \subsubsection{Formato}
    
  \rule{\linewidth}{1pt}

    \paragraph{Llaves}
      Siempre se deben utilizar llaves en las instrucciones \textit{if}, \textit{else}, 
      \textit{do} y \textit{while}, incluso si no son necesarios.

      Esto quiere decir que el siguiente código:
      
      \begin{minted}{java}
        if (a >= 0) return a;
        else return -a;
      \end{minted}

      debe ser cambiado por:

      \begin{minted}{java}
        if (a >= 0) {
          return a;
        } else {
          return -a;
        }
      \end{minted}
    %
    \paragraph{Indentación}
      Las instrucciones deben tener una indentación de 2 espacios.
    %
    \paragraph{Una instrucción por línea}
      Cada instrucción debe estar seguida de un salto de línea.

      Esto quiere decir que:

      \begin{minted}{java}
        // Esto
        int i = 0; double d = 1.0;
        // Se debe cambiar por
        int i = 0;
        double d = 1.0;
      \end{minted}
    %
    \paragraph{Límite de columnas}
      Cada línea no debe exceder los 100 caracteres.
    %
    \paragraph{Espacios horizontales}
      \begin{itemize}
        \item Separar cada \textit{keyword} de los paréntesis que lo sigan en esa línea, 
          e.g. \mintinline{java}{if (...)} en vez de \mintinline{java}{if(...)}.
        %
        \item Separar cada \textit{keyword} de la llave que lo precede, e.g. 
          \mintinline{java}!} else! en vez de \mintinline{java}!}else!.
        %
        \item Colocar un espacio antes de cada apertura de llave (e.g. 
          \mintinline{java}{if (...) {}} en lugar de \mintinline{java}{if (...){}}), con 2 
          excepciones:

          \begin{itemize}
            \item \mintinline{java}{@SomeAnnotation({a, b})}.
            \item Al definir arreglos como \mintinline{java}{new int[][] {{1, 2}}}.
          \end{itemize}
        %
        \item En ambos lados de un operador binario o ternario, e.g. 
          \mintinline{java}{c = a + b} en vez de \mintinline{java}{c=a+b}.
        %
        \item Luego de \texttt{,}, \texttt{:} y \texttt{;}, y luego del cierre de un 
          paréntesis al hacer \textit{casting}, e.g. 
          \mintinline{java}{for (int i = 0; i < j; i++) {}}, \mintinline{java}{int a, b},
          \mintinline{java}{a > b ? a : b}, \mintinline{java}{a = (int) 2.5}.
      \end{itemize}
    %
  %
  \subsubsection{Nombres}
    \rule{\linewidth}{1pt}

    Los nombres en todos los casos siguientes deben ser descriptivos.
    
    \begin{minted}{java}
      // Evitar esto
      int c = 0; // Product stock counter
      // y reemplazarlo por
      int productStock = 0;
    \end{minted}

    \paragraph{Paquetes}
      Los nombres de paquetes deben contener solo letras minúsculas, e.g. 
      \mintinline{java}{package com.github.cc3002metodologias} en lugar de
      \mintinline{java}{package com.gitHub.CC3002Metodologias}.
    %
    \paragraph{Clases}
      Los nombres de las clases utilizan \textit{UpperCamelCase} y deben ser sustantivos.
    %
    \paragraph{Métodos}
      Los nombres de métodos se escriben en \textit{lowerCamelCase} y deben ser verbos o 
      acciones.
    %
    \paragraph{Variables \texttt{static final}}
      Los nombres de estas variables utilizan \textit{CONSTANT\_CASE}.
    %
    \paragraph{Otras variables}
      
    %
  %
%