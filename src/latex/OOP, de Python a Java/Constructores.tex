% !TEX root = D:\Users\Ignacio\Documentos\Escuela\CC3002 - Metodologías de Diseño y Programación\apunte-y-ejercicios\src\latex\Apunte.tex
\section{Constructores}
  
  En la sección anterior creamos una clase \texttt{Vector2D} que se inicializaba con 2
  parámetros.
  El método que se encarga de crear una instancia de una clase se llama constructor.

  El constructor es un método especial que no tiene tipo de retorno (esto es distinto a 
  que retorne \mintinline{java}{void}) y que se llama al momento de crear una instancia de
  una clase.

  En \textit{Python} el constructor de una clase se define como 
  \mintinline{python}{def __init__(self, param1, param2, ...):}, mientras que en 
  \textit{Java} se hace creando un método con la firma 
  \mintinline{java}{ClassName(Param1Type param1, Param2Type param2, ...) {...}}.
  En el caso de \textit{Java}, el método constructor \textbf{debe llamarse igual que la 
  clase} a la que pertenece.

  Por defecto en \textit{Python} y \textit{Java}, si no se define un constructor 
  explícitamente, las clases tienen un constructor vacío que se crea implícitamente y lo 
  único que hace es reservar un espacio en memoria para el objeto que se vaya a crear, 
  esto es lo mismo que si se definiera una clase como:

  \begin{listing}[ht!]
    \begin{minted}{java}
      class ClassName {
        ClassName() {}
      }
    \end{minted}
  \end{listing}

  \begin{important}
    Si se define un constructor de forma explícita entonces no se crea el constructor por 
    defecto.
  \end{important}

  El objetivo de los constructores es definir las condiciones iniciales de un objeto, 
  generalmente esto significa iniciar las variables de instancia.

  Supongan ahora que queremos crear usuarios para alguna aplicación.
  Todos los usuarios deben tener un nombre que los identifique y opcionalmente pueden 
  ingresar su género.
  ¿Cómo manejamos la existencia de parámetros opcionales?

  En \textit{Python} se pueden utilizar parámetros por defecto, por lo que se podría 
  implementar esto como:

  \begin{listing}[ht!]
    \begin{minted}{python}
      class User:
          def __init__(self, username, gender="Non specified"):
              self.username = username
              self.gender = gender
    \end{minted}
  \end{listing}

  Esto no es posible de hacer en \textit{Java} ya que no existen parámetros por defecto.
  En vez de eso lo que se puede hacer (y que no se puede hacer en \textit{Python}) es 
  crear múltiples constructores, entonces:

  \begin{listing}[ht!]
    \begin{minted}{java}
      class User {
        String username, gender;

        User(String username, String gender) {
          this.username = username;
          this.gender = gender;
        }

        User(String username) {
          this(username, "Non specified");
        }
      }
    \end{minted}
  \end{listing}

  En el próximo capítulo profundizaremos en el funcionamiento de \textit{Java} y 
  explicaremos cómo funciona \mintinline{java}{this}, pero por ahora lo que necesitan 
  saber es que \mintinline{java}{this(username, "Non specified")} llama al otro 
  constructor con esos parámetros.

  \begin{exercise}
    Existen lenguajes de programación como \textit{Smalltalk} que no tienen constructores,
    sino que los objetos se crean vacíos y los parámetros se inician luego con un método
    \enquote{normal}.

    ¿Por qué es conveniente que los constructores sean métodos especiales?
    Discuta
  \end{exercise}

  \subsection{Casos especiales}
    Existen objetos especiales que pueden iniciarse sin necesidad de llamar explícitamente
    al constructor.
    En \textit{Java} este es el caso de las clases \texttt{Array} y \texttt{String}.

    El caso de los \textit{strings} es el más típico ya que cuando creamos un objeto de 
    tipo \mintinline{java}{String} no hacemos \mintinline{java}{new String(...)} (aunque 
    se puede), sino que simplemente escribimos texto entre comillas dobles.
    Es importante notar en este punto que los \textit{strings} se definen entre comillas
    dobles y los caracteres entre comillas simples, entonces \mintinline{java}{'a'} es un
    primitivo pero \mintinline{java}{"a"} es un objeto.
    
    El otro caso son los arreglos, a diferencia de los \textit{strings}, estos sí se crean
    utilizando la \textit{keyword} \mintinline{java}{new}, pero no se llama explícitamente al 
    constructor.
    En cambio, existen dos maneras de iniciar un arreglo.

    Los arreglos tienen un \textbf{tamaño fijo} (un arreglo no es lo mismo que una lista), 
    por lo que es necesario definir ese tamaño al momento de crearlo.
    La sintaxis para hacer esto es \mintinline{java}{T[] array = new T[size]}, esto va a 
    crear un arreglo de tamaño \texttt{size} de elementos de tipo \texttt{T} lleno con 
    valores nulos o, en caso de ser un tipo primitivo, con sus valores por defecto.
    Por ejemplo:

    \begin{listing}[ht!]
      \begin{minted}{java}
        int[] ints = new int[3];
        ints[0] = 100;
        ints[2] = 11;
        System.out.println(ints[0]); // Imprime 100
        System.out.println(ints[1]); // Imprime 0
      \end{minted}
    \end{listing}

    La otra manera de crear un arreglo es pasándole inmediatamente sus valores, la 
    sintaxis es similar a la anterior con la diferencia de que no se define el tamaño 
    (porque éste se infiere por la cantidad de valores).
    Así, el mismo arreglo del ejemplo anterior se puede crear de la siguiente forma:

    \begin{listing}[ht!]
      \begin{minted}{java}
        int[] ints = new int[] {100, 0, 11};
      \end{minted}
    \end{listing}

    \begin{exercise}
      Cree una clase \texttt{Figure2D} con un método \mintinline{java}{printType()}
      que imprima el tipo de la figura.
      Para esto, considere una figura como un conjunto de puntos (de la clase 
      \mintinline{java}{Vector2D}), donde \mintinline{java}{printType()} debe indicar que
      la figura es un punto si ésta tiene un solo punto, si tiene 2 puntos entonces dice 
      que es una línea y si tiene 3 o más entonces es un polígono.

      Resuelva este ejercicio utilizando solo constructores, y variables de instancia de 
      tipo primitivo, \mintinline{java}{Vector2D}, \textit{strings} y arreglos, y sin
      utilizar métodos adicionales (tampoco los de las clases \texttt{String} o 
      \texttt{Array}) ni instrucciones de control de flujo (como 
      \mintinline{java}{if| o |switch}).

      \textit{Hint: Utilice varios constructores y \textbf{al menos} una variable de 
      instancia.}
    \end{exercise}
  %
%