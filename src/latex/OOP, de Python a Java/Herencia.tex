% !TEX root = D:\Users\Ignacio\Documentos\Escuela\CC3002 - Metodologías de Diseño y Programación\apunte-y-ejercicios\src\latex\Apunte.tex
\section{Herencia}

  En las secciones anteriores dimos una definición para un punto en 2 dimensiones, pero un 
  punto en un plano es algo bastante limitado.
  En esta sección tomaremos la definición inicial de nuestro punto y la generalizaremos 
  para luego, en base a nuestra nueva definición, crear casos más específicos.

  Consideren un vector como una línea que va desde el origen del sistema de coordenadas 
  hasta un punto.
  Un vector en un espacio euclídeo de dimensión \(n\) es una n-tupla de números reales.
  Dada esta definición podemos definir una clase \texttt{VectorND} de la siguiente 
  forma:\footnote{
    La implementación en \textit{Python} la pueden encontrar 
    \href{https://github.com/CC3002-Metodologias/apunte-y-ejercicios/blob/master/src/main/python/vectors.py}{aquí}
  }
  
  \begin{listing}[ht!]
    \begin{minted}{java}
      class VectorND {
        double[] tail;
        
        VectorND(double[] tail) {
          this.tail = tail;
        }
      }
    \end{minted}
  \end{listing}

  Ahora, una característica de los vectores es que pueden sumarse entre ellos.
  La suma de dos vectores es sólo la suma de sus coordenadas, teniendo en cuenta las 
  dimensiones de estos (un vector de \(m\) dimensiones, con \(m \leq n\) es un vector de 
  \(n\) dimensiones en el que todas las componentes \(\mathbf{v}_i\) para \(i > m\) son 
  0).

  \begin{minted}{java}
    VectorND add(VectorND otherVector) {
      double[] bigger, smaller;
      if (tail.length > otherVector.tail.length) {
        bigger = tail;
        smaller = otherVector.tail;
      } else {
        bigger = otherVector.tail;
        smaller = tail;
      }
      double[] components = new double[bigger.length];
      for (int i = 0; i < smaller.length; i++) {
        components[i] = bigger[i] + smaller[i];
      }
      for (int i = smaller.length; i < bigger.length; i++) {
        components[i] = bigger[i];
      }
      return new VectorND(components);
    }
  \end{minted}

  Definamos además un método \texttt{print} que imprima el vector de la forma \((x_0, x_1, 
  ..., x_{n - 1})\).

  
  \begin{minted}{java}
    void print() {
      String result = "(";
      for (int idx = 0; idx < tail.length; idx++) {
        result += tail[idx];
        if (idx < tail.length - 1) {
          result += ", ";
        }
      }
      System.out.println(result + ")");
    }
  \end{minted}
  
  Ahora, es poco común trabajar con vectores de \(n\) dimensiones, en general trabajaremos
  con vectores de dos o tres dimensiones, así que sería una buena idea tener clases 
  específicas para dichos tipos.
  Ahora, ¿Por qué es una buena idea?

  Como mencionamos en el capítulo \ref{ch:oop}, el objetivo de la herencia es la 
  \textbf{especialización} de una clase, y eso es precisamente lo que queremos hacer aquí.
  Tomar un vector de \(n\) dimensiones y crear casos específicos para otros con 
  dimensiones fijas (que tendrán propiedades propias de acuerdo a sus dimensiones).

  Cambiemos el nombre de nuestra clase \texttt{Vector2D} por \texttt{Vector2D} y hagamos
  que sea una subclase de \texttt{VectorND}.

  En \textit{Python}, esto se haría de la siguiente forma:

  \begin{minted}{python}
    class Vector2D(VectorND):
        def __init__(self, x, y):
            super().__init__([x, y])
  \end{minted}

  Aquí, la clase entre paréntesis es la superclase de \texttt{Vector2D}.
  La línea \mintinline{python}{super().__init__([x, y])} lo que hace es llamar al 
  constructor de la superclase (\texttt{VectorND}) y crear un objeto de tipo 
  \texttt{Vector2D} con la lista que se le entregan como parámetros.

  En el caso de \textit{Java}, la herencia se define con la \textit{keyword} 
  \mintinline{java}{extends} de la siguiente forma:

  \begin{minted}{java}
    class Vector2D extends VectorND {

      Vector2D(double x, double y) {
        super(new double[] {x, y});
      }
    }
  \end{minted}

  Como pueden ver, la sintaxis con la que se llama al constructor de la superclase es 
  similar a la manera en la que se hace en \textit{Python}.

  Ahora, el método \texttt{add} que habíamos definido anteriormente en nuestra clase 
  \texttt{Point2D} ya no es necesario.
  ?`Por qué pasa esto?
  La explicación la dimos en el capítulo \ref{ch:oop}, ahí dijimos que los hijos heredan 
  sus funcionalidades de su padre, así que si borramos el método \texttt{add} de nuestra
  clase \texttt{Vector2D}, entonces se llamará al método de la superclase.
  Entonces, deberíamos poder ejecutar el método \texttt{add} de la misma forma que 
  habíamos hecho antes en la sección \ref{sec:methods} y obtener el mismo resultado.

  \begin{minted}{java}
    Vector2D point = new Vector2D(1, 3);
    Vector2D newPoint = point.add(new Vector2D(-1, 2));
  \end{minted}

  Si intentan correr el código anterior se darán cuenta que no compila.
  

  \begin{note}
    Un detalle que vale la pena mencionar es que \textbf{todos los objetos}, tanto en 
    \textit{Python} como \textit{Java}, extienden implícitamente a un objeto especial 
    llamado \texttt{object} en \textit{Python} y \texttt{Object} en \textit{Java}.
    Así, la definición de la clase \texttt{VectorND} podría hacerse como 
    \mintinline{java}{class VectorND extends Object {...}} y sería equivalente 
    (en \textit{Python} es análogo).
  \end{note}
%