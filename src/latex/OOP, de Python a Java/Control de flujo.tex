% !TEX root = D:\Users\Ignacio\Documentos\Escuela\CC3002 - Metodologías de Diseño y Programación\apunte-y-ejercicios\src\latex\Apunte.tex
\section{Control de flujo}
  En programación, se le llama \textit{control de flujo} a las instrucciones que 
  dependerán de una condición para ``decidir'' qué acción debe realizar a continuación el 
  programa, por ejemplo, las instrucciones \textit{if-else} o \textit{while}.

  Comencemos por la más simple, implementar una instrucción \textit{if-else} en 
  \textit{Python} se haría de la siguiente forma:

  \begin{minted}{python}
    if a > b:
        print("a > b")
    elif a == b:
        print("a = b")
    else:
        print("a > b")
  \end{minted}

  En este caso, el inicio y fin de cada rama del bloque \textit{if-else} depende de la 
  indentación, ya mencionamos que esta no es importante en \textit{Java}.
  El siguiente código es equivalente al anterior:

  \begin{minted}{java}
    if (a > b) {
      System.out.println("a > b");
    } else if (a == b) {
      System.out.println("a == b");
    } else {
      System.out.println("a < b");
    }
  \end{minted}

  Ahora, veamos la instrucción \textit{while}.
  En \textit{Python}:

  \begin{minted}{python}
    while i > 0:
      print(i)
      i -= 1  # Equivalente a hacer i = i - 1
  \end{minted}

  Lo que en \textit{Java} se podría hacer como:

  \begin{minted}{java}
    while (i > 0) {
      System.out.println(i);
      i -= 1;
    }
  \end{minted}

  El siguiente código es equivalente al anterior:
  
  \begin{minted}{java}
    while (i > 0) {
      System.out.println(i--);
    }  
  \end{minted}

  Análogamente, también se puede hacer \mintinline{java}{i++}.

  \begin{exercise}
    Pruebe cambiar la expresión \mintinline{java}{i--} por \mintinline{java}{--i}, los 
    resultados son distintos. 
    ¿Por qué pasa esto?
  \end{exercise}

  \textit{Java} además tiene otra manera de implementar un \textit{loop} conocida como 
  \textit{do-while}, esta funciona exactamente igual que una expresión \textit{while} 
  común con la diferencia de que \textbf{siempre} el bloque dentro del \textit{while} se
  ejecutará al menos 1 vez.
  El mismo ejemplo anterior:

  \begin{listing}[ht!]
    \begin{minted}{java}
      do {
        System.out.println(i--);
      } while (i > 0);
    \end{minted}
  \end{listing}

  Otra instrucción de control de flujo típica es \textit{for}. 
  En \textit{Python} podemos hacer:

  \begin{listing}[ht!]
    \begin{minted}{python}
      for i in range(0, 5):
          print(i)
    \end{minted}
  \end{listing}
  
  En \textit{Java} podemos hacer lo mismo como:

  \begin{listing}[ht!]
    \begin{minted}{java}
      for (int i = 0; i < 5; i++) {
        System.out.println(i);
      }
    \end{minted}
  \end{listing}

  Expliquemos un poco esta sintaxis.
  La primera parte del \textit{for} indica él o los valores iniciales del \textit{loop} 
  (en este caso define una variable \texttt{i} con el valor 0), la segunda parte es la 
  condición que se debe cumplir para que se ejecute el código dentro del bloque (aquí, se 
  imprimirá en consola mientras \texttt{i} sea menor a 5), y, por último, la tercera parte
  es la instrucción que se ejecutará cada vez que termine de ejecutarse el bloque (para el 
  ejemplo, en cada iteración se le suma 1 a \texttt{i}).

  Una de las características más útiles de la instrucción \textit{for} es la capacidad de
  iterar sobre una estructura de datos, un ejemplo podría ser:
  
  \begin{listing}[ht!]
    \begin{minted}{python}
      pairs = [0, 2, 4, 6, 8]
      for i in pairs:
          print(i)
    \end{minted}    
  \end{listing}

  En \textit{Java} podemos hacer algo equivalente de la forma:

  \begin{listing}[ht!]
    \begin{minted}{java}
      int[] pairs = new int[]{0, 2, 4, 6, 8};
      for (int i : pairs) {
        System.out.println(i);
      }
    \end{minted}
  \end{listing}

  
  Adicionalmente, en \textit{Java} existe otra instrucción de control de flujo que no está
  presente en \textit{Python}, esta se conoce como \textit{switch-case} y es parecido a 
  hacer un \textit{if-else} de una manera más compacta, pero que tiene más restricciones.
  Un ejemplo de esta instrucción sería:

  \begin{listing}[ht!]
    \begin{minted}{java}
      switch (a) {
        case 0:
          System.out.println("a = 0");
          break;
        case 1:
          System.out.println("a = 1");
          break;
        case 2:
          System.out.println("a = 2");
          break;
        default:
          System.out.println("a is not in [0, 2]");
      }
    \end{minted}
  \end{listing}

  \begin{exercise}
    Borre las instrucciones \mintinline{java}{break} del ejemplo anterior y compruebe si 
    el comportamiento es el mismo.
  \end{exercise}
%