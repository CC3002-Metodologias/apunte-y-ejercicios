\section{\textit{Input}}
  En la sección anterior mostramos un programa que era capaz de imprimir un mensaje en
  pantalla, pero siempre que lo ejecutemos hará lo mismo.
  ¿Qué hacemos para que el programa reciba \textit{input} de un usuario?
  Para esto tendremos dos opciones:

  \subsection{Opción 1: Argumentos por consola}
    La primera opción es la que usan la mayoría de aplicaciones de consola que vienen 
    integradas en los sistemas operativos, i.e. recibir los parámetros como argumentos
    entregados al momento de ejecutar el programa, por ejemplo:

    \begin{minted}{bash}
      cd path/to/dir
    \end{minted}

    Modifiquemos un poco el ejemplo anterior para que reciba parámetros desde consola,
    en el caso de \textit{Python}, para recibir los argumentos se debe hacer una llamada
    al sistema, el siguiente ejemplo muestra como hacer esto:

    \begin{minted}{python}
      import sys

      if __name__ == "__main__":
          # Tomamos todos los argumentos menos el primero porque en Python el primer 
          # argumento siempre es el nombre del archivo.
          name = " ".join(sys.argv[1:])
          print(f"My name is {name}, and I have a dream.")
    \end{minted}

    Y luego podemos ejecutarlo como:

    \begin{minted}{bash}
      python3 ejemplo_basico.py Ignacio Slater
    \end{minted}

    Por otro lado, en \textit{Java} tendríamos:

    \begin{minted}{java}
      public class EjemploBasico {

        public static void main(String[] args) {
          System.out.println("My name is " + String.join(" ", args) + ", and I have a dream.");
        }
      }
    \end{minted}

    Como pueden ver, en este caso no hay necesidad de hacer una llamada explícita al 
    sistema para obtener los argumentos, ya que el método \texttt{main} lo hace por 
    defecto.
    De manera similar a lo que hicimos para ejecutar el programa en \textit{Python}, 
    ahora ejecutaremos éste como:

    \begin{minted}{bash}
      javac EjemploBasico.java
      java EjemploBasico Ignacio Slater
    \end{minted}      

    En ambos casos el resultado debiera ser el mismo.
  %

  \subsection{Opción 2: Pedir parámetros interactivamente}
    La otra opción es hacer que el usuario ingrese parámetros durante la ejecución del 
    programa.

    En \textit{Python}, si quisiéramos hacer eso, tendríamos que modificar el programa 
    anterior como:

    \begin{minted}{python}
      if __name__ == "__main__":
          name = input("Write your name: ")
          print(f"My name is {name}, and I have a dream.")
    \end{minted}

    Nuevamente, en \textit{Java} el código sería más extenso:

    \begin{minted}{java}
      import java.util.Scanner;

      public class EjemploBasico {

        public static void main(String[] args) {
          System.out.println("Write your name: ");
          String name = new Scanner(System.in).nextLine();
          System.out.println("My name is " + name + ", and I have a dream.");
        }
      }
    \end{minted}

    No entraremos en más detalles dentro de estos conceptos ya que no serán utilizados 
    durante el curso.

    \begin{exercise}
      Escriba un programa en \textit{Java} que reciba interactivamente texto desde consola
      e imprima de vuelta el mensaje entregado, hasta que se ingrese una línea vacía.
      
      Para comprobar si un texto está vacío, pueden hacerlo como 
      \mintinline{java}{text.isEmpty()}, de manera similar, si quieren comprobar que no 
      está vacío se hace como \mintinline{java}{!text.isEmpty()}
    \end{exercise}
  %
%