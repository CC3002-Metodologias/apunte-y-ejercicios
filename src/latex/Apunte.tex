\documentclass[12pt]{book}
\usepackage[spanish]{babel}
\usepackage{authblk}
\usepackage{fullpage}

\title{CC3002 - Metodologías de diseño y programación}

\author{Juan-Pablo Silva}
\author{Ignacio Slater}
\affil{Departamento de Ciencias de la Computación, Universidad de Chile}

\date{\today}

\begin{document}
  \frontmatter
  \maketitle
  \tableofcontents
  \chapter{Introducción}
  %

  \mainmatter
  \part{Objetos en Java}
    \chapter{Programación orientada a objetos}
      Hasta el momento, gran parte de lo que ustedes conocen es cómo escribir algoritmos 
      en los que realizan acciones siguiendo una lógica.
      La programación orientada a objetos (OOP) es un \textbf{paradigma} de computación 
      que se organiza en base a \textbf{objetos} en vez de acciones y \textbf{datos} en 
      lugar de lógica.

      Esto requiere un gran cambio de enfoque respecto a la programación imperativa 
      tradicional que están acostumbrados a usar puesto que el enfoque estará en cuáles 
      son los objetos que vamos a manipular en vez de la lógica para manipularlos.
    %
  %
  \part{Patrones y metodologías de diseño}
  %
\end{document}