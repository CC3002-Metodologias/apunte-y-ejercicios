% !TEX root = C:\Users\smfla\Documents\Escuela\CC3002 - Metodologías de Diseño y Programación\apunte-y-ejercicios\src\latex\Apunte.tex
\section{Contextos estáticos}
  Hasta ahora hemos estado utilizando la keyword \mintinline{java}{static} para escribir nuestros
  programas, pero no nos hemos detenido a ver lo que significa.
  ¿Cuál es la diferencia entre declarar \mintinline{java}{public static void main(String[] args)} y
  \mintinline{java}{public void main(String[] args)}?
  El objetivo de esta sección será explicar a grandes rasgos qué son y cómo se utilizan los 
  \textit{contextos estáticos}.

  % region : Métodos
  \subsection{Métodos estáticos}
    Comencemos con un ejemplo. 
    Siguiendo con lo que hemos construido hasta ahora, supongan que ahora queremos implementar una 
    poción que cure 10 \textit{hit points} a un \textit{Bakémon}.
    Un primer acercamiento a este problema podría ser:

    \begin{minted}{java}
      public class Potion {
        public void useOn(Bakemon target) {
          target.setHitPoints(target.getHitPoints() + 10);
        }
      }
    \end{minted}

    Ahora, el código anterior puede ser un poco confuso de leer, así que agreguemos un método auxiliar
    en nuestra clase \texttt{Bakemon} para simplificar nuestro código.

    \begin{minted}{java}
      // com.github.cc3002.bucket.monsters.Bakemon
      public void increaseHitPoints(int amount) {
        this.hitPoints += amount;
      }
    \end{minted}

    Con este cambio podemos simplificar nuestra implementación anterior.

    \begin{minted}{java}
      // com.github.cc3002.bucket.items.Potion
      public void useOn(Bakemon target) {
        target.increaseHitPointsBy(10);
      }
    \end{minted}

    Utilicemos lo que llevamos, creemos un \textit{Bakémon}, disminuyamos su vida y luego curémoslo
    con la poción.

    \begin{minted}{java}
      Bakemon italicore = new Bakemon("Italicore", 15);
      Potion potion = new Potion();
      italicore.setHitPoints(0);
      potion.useOn(italicore);
      System.out.println("hp = " + italicore.getHitPoints());
    \end{minted}

    El código anterior funciona como esperamos, pero tiene un problema que a pequeña escala podría 
    parecer irrelevante, pero en proyectos grandes podría generar problemas graves.

    Cada vez que creo una poción, ese objeto ocupa espacio en la memoria que permanece ahí durante un 
    tiempo que no podemos definir.
    ¿Qué pasa si tenemos 100 pociones? ¿y 1000?
    Más aún, cuando definimos los objetos en el capítulo \ref{ch:oop} dijimos que un objeto se 
    componía de dos cosas (un comportamiento y un estado), pero este objeto no tiene estado.
    ¿Tiene sentido entonces que existan instancias de la clase \texttt{Potion}?

    Nos gustaría poder definir esto de manera más eficiente en memoria y, de paso, mejorar el diseño 
    de nuestra solución.
    Es aquí donde entran los métodos estáticos.

    Un \textbf{método estático} es un método que no participa del \textit{method lookup} ya que va 
    asociado a la clase y no a las implementaciones de esta.
    Esto se parece a los constructores, pero existen ciertas diferencias, pero antes de ver estas 
    diferencias veamos cómo modificar nuestra clase original para solucionar el error que habíamos 
    cometido.

    Para definir un método estático, solamente necesitamos utilizar la \textit{keyword} 
    \mintinline{java}{static} en la definición del método.

    \begin{minted}{java}
      // com.github.cc3002.bucket.items.Potion
      public static void useOn(Bakemon target) {
        target.increaseHitPointsBy(10);
      }
    \end{minted}

    Este pequeño cambio tiene un gran impacto en la eficiencia del programa y la diferencia tanto en
    su implementación y su uso son mínimos.
    Modificando el ejemplo de antes para utilizar el método estático tendríamos:
    \begin{minted}{java}
      Bakemon italicore = new Bakemon("Italicore", 15);
      italicore.setHitPoints(0);
      Potion.useOn(italicore);
      System.out.println("hp = " + italicore.getHitPoints());
    \end{minted}

    Noten que ahora, como el método va asociado a la clase, no instanciamos el objeto para llamar el
    método.
    
    Tanto los constructores como los métodos estáticos van ligados a la clase y no a instancias de
    ésta.
    ¿Entonces qué los diferencia?
    Si recuerdan la sección anterior, entonces ya lo saben.
    Los métodos estáticos son (valga la redundancia) métodos, mientras que los constructores no.
    ¿Qué implicancias tiene esto?

    La primera consecuencia que se puede notar de esto es que las instancias de una clase pueden
    acceder a los métodos estáticos de ésta (después de todo son instancias \textbf{de} la clase),
    pero esto no se puede hacer en la dirección contraria.
    Esto último también implica que no se pueden llamar a métodos estáticos desde métodos de 
    instancia de la clase.
    Un ejemplo de esto último sería intentar hacer
    
    \begin{minted}{java}
      // com.github.cc3002.bucket.items.Potion
      public void useTwiceOn(Bakemon target) {
        this.useOn(target)
      }
    \end{minted}

    Pueden probar correr el último método y ver que no funciona.

    \begin{exercise}
      ¿Se pueden heredar métodos estáticos?
      ¿A qué se debe esto último?
    \end{exercise}
    
    \begin{exercise}
      Dadas las definiciones anterior. 
      ¿Puede dar una razón sobre por qué el método \texttt{main} debe ser estático?
    \end{exercise}
  % endregion
  %% region : Variables
  \subsection{Variables estáticas}
    Adicionalmente a los métodos estáticos también existen las variables estáticas.
    Estas variables funcionan de la misma manera que los métodos estáticos ya que tampoco participan
    del \textit{method lookup}.
    La diferencia con los métodos es que al ser variables pueden asignarse y accederse a sus valores
    de la misma forma que cualquier otra variable de la clase.

    Un ejemplo de uso de variable estática es \texttt{Math.PI}, noten que esto tiene sentido ya que
    \(\pi\) es una constante y por lo tanto es independiente del contexto en el que se le llame 
    \footnote{Volveremos a visitar esta clase en la sección (véase sección \ref{sec:final})}.

    Retomemos el ejemplo de la poción para mostrar cómo utilizar estas variables.

    \begin{minted}{java}
      public class Potion {
        public static int healPoints = 10;

        public static void useOn(Bakemon target) {
          target.increaseHitPointsBy(healPoints);
        }
      }
    \end{minted}

    Luego, podemos usar la clase como:
    \begin{minted}{java}
      Bakemon italicore = new Bakemon("Italicore", 15);
      italicore.setHitPoints(0);
      Potion.useOn(italicore);
      System.out.println("hp = " + italicore.getHitPoints());
      italicore.setHitPoints(0);
      Potion.healPoints = 15;
      Potion.useOn(italicore);
      System.out.println("hp = " + italicore.getHitPoints());
    \end{minted}

    Las variables estáticas suelen utilizarse para definir constantes, ya que el valor será el mismo
    para todas las instancias de la clase, pero este mismo comportamiento puede llevar a errores muy
    difíciles de encontrar.
    Veamos un pequeño ejemplo.
  
    \begin{minted}{java}
      public class Example {
        private int i = 1;

        public Example() {
          System.out.println(++i);
        }

        public static void main(String[] args) {
          new Example();
          new Example();
        }
      }
    \end{minted}

    Si ejecutamos el código anterior el programa imprimiría en consola
    \begin{minted}{text}
      2
      2
    \end{minted}

    Pero si cambiamos la variable \texttt{i} por una estática el resultado es totalmente distinto,
    si declaramos la variable como \mintinline{java}{private static int i = 1} el programa va a 
    imprimir
    \begin{minted}{text}
      2
      3
    \end{minted}
    
    ¿Por qué pasa esto?
    Ya mencionamos la razón, las variables estáticas van ligadas a la clase y no a instancias 
    puntuales de ésta, o dicho de otra forma, son comunes para todas las instancias de la clase.

    Existen otros usos y tipos de contextos estáticos, pero son menos comunes y no serán de 
    importancia en el curso, así que serán omitidos.

    \begin{exercise}
      ¿Qué sucede si cambiamos el ejemplo de la poción por lo siguiente?
      \begin{minted}{java}
        public class Potion {
          public int healPoints = 10;
  
          public static void useOn(Bakemon target) {
            target.increaseHitPointsBy(healPoints);
          }
        }
      \end{minted}

      ¿Por qué ocurre esto?
    \end{exercise}

    \begin{exercise}
      ¿Se pueden heredar variables estáticas?
      ¿Por qué?
    \end{exercise}

    \begin{exercise}
      ¿Se pueden llamar métodos o variables estáticas desde el constructor?
    \end{exercise}
  % endregion