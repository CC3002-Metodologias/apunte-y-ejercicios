% !TEX root = D:\Users\Ignacio\Documentos\Escuela\CC3002 - Metodologías de Diseño y Programación\apunte-y-ejercicios\src\latex\Apunte.tex
\chapter{¿\textit{Java}?}
  \label{ch:java}
  Este capítulo servirá como continuación sobre programación orientada a objetos, pero 
  utilizando las herramientas que provee \textit{Java} y la manera en que este lenguaje se
  comporta y las herramientas básicas que serán utilizadas durante el curso.

  Los contenidos de este capítulo son \textbf{específicos a \textit{Java}} y no 
  necesariamente son iguales o análogos a los de otros lenguajes.

  Cabe destacar también que son contenidos muy básicos sobre el funcionamiento y las 
  herramientas que proporciona éste.
  En el capítulo \ref{ch:java-2} se verán algunos conceptos más avanzados, pero aún así
  debe considerar que \textit{Java} es mucho más complejo que lo que cubrirá este apunte.

  Además, en este capítulo (ya que los ejemplos serán más complejos) se comenzará a 
  utilizar \textit{IntelliJ IDEA} como herramienta para programar, compilar y ejecutar 
  el código que se escriba.

  \subimport{Profundizando en Java/}{Convenciones.tex}
  \subimport{Profundizando en Java/}{Estructura de una aplicación.tex}
  \subimport{Profundizando en Java/}{Visibilidad.tex}
  \section{\textit{Method lookup}}
    \label{sec:lookup}
  %
  \subimport{Profundizando en Java/}{Constructores.tex}
  \subimport{Profundizando en Java/}{Métodos estáticos.tex}
  \subimport{Profundizando en Java/}{Clases abstractas.tex}
  \subimport{Profundizando en Java/}{libreria_estandar.tex}

  \nocite{*}
  \printbibliography[keyword=java-2]
%