% !TEX root = D:\Users\Ignacio\Documentos\Escuela\CC3002 - Metodologías de Diseño y Programación\apunte-y-ejercicios\src\latex\Apunte.tex
\section{\textit{Windows}}
  \subsection{Opción 1: \textit{OpenJDK} con \textit{Chocolatey} (Recomendado)}
    Si no lo tienen instalado, el primer paso sería instalar el gestor de paquetes 
    \textit{Chocolatey}, para esto se debe abrir \textit{Powershell} como 
    administrador.

    \begin{minted}{powershell}
      [Net.ServicePointManager]::SecurityProtocol = `
        [Net.SecurityProtocolType]::Tls12
      Set-ExecutionPolicy -ExecutionPolicy RemoteSigned -Force
      Invoke-WebRequest "https://chocolatey.org/install.ps1" -UseBasicParsing `
        | Invoke-Expression
    \end{minted}

    Una vez que tengan \textit{Chocolatey} instalado, basta ejecutar:

    \begin{minted}{powershell}
      cinst openjdk -y
    \end{minted}

    Para probar que la instalación se haya hecho de forma correcta, cierren y abran 
    \textit{Powershell} y ejecuten el comando:

    \begin{minted}{powershell}
      java -version
    \end{minted}

    El input debiera ser algo como:

    \begin{minted}{text}
      openjdk version "14" 2020-03-17
      OpenJDK Runtime Environment (build 14+36-1461)
      OpenJDK 64-Bit Server VM (build 14+36-1461, mixed mode, sharing)
    \end{minted}

    Si esto no funciona, proceda con la siguiente opción.
  %

  \subsection{Opción 2: \textit{OpenJDK} sin \textit{Chocolatey}}
    Primero deben descargar los binarios del \textit{JDK} desde 
    \href{
      https://download.java.net/java/GA/jdk14/076bab302c7b4508975440c56f6cc26a/36/GPL/openjdk-14_windows-x64_bin.zip
      }{aquí}.
    
    Con los binarios descargados, extraigan el \texttt{.zip} en algún directorio y luego 
    abran \textit{Powershell} como administrador y ubíquense en la carpeta donde 
    extrajeron el archivo.
    Una vez ahí, ejecuten:

    \begin{minted}{powershell}
      Move-Item -Path .\jdk-14 -Destination $Env:Programfiles
      [Environment]::SetEnvironmentVariable("JAVA_HOME", 
                                            "$Env:Programfiles\jdk-14")
      [Environment]::SetEnvironmentVariable(
        "Path", 
        [Environment]::GetEnvironmentVariable('Path', 
        [EnvironmentVariableTarget]::Machine) + ";$($Env:JAVA_HOME)\bin", 
        [EnvironmentVariableTarget]::Machine)
    \end{minted}

    Para probar que la instalación se haya hecho de forma correcta, cierren y abran 
    \textit{Powershell} y ejecuten el comando:

    \begin{minted}{powershell}
      java -version
    \end{minted}

    El input debiera ser algo como:

    \begin{minted}{text}
      openjdk version "14" 2020-03-17
      OpenJDK Runtime Environment (build 14+36-1461)
      OpenJDK 64-Bit Server VM (build 14+36-1461, mixed mode, sharing)
    \end{minted}

    Si esto no funciona, proceda con la siguiente opción.
  %

  \subsection{Opción 3: \textit{Oracle JDK}}
    Lo primero es descargar el \textit{JDK} desde el 
    \href{
      https://www.oracle.com/java/technologies/javase-jdk14-downloads.html
    }{sitio de \textit{Oracle}} y una vez descargado ejecuten el instalador y sigan las 
    instrucciones.

    Por último, deben agregar el \textit{path} de \textit{Java} a las variables de 
    entorno, para esto abran \textit{Powershell} como administrador y ejecuten:

    \begin{minted}{powershell}
      [Environment]::SetEnvironmentVariable("JAVA_HOME", 
                                            "$Env:Programfiles\Java\jdk-14")
      [Environment]::SetEnvironmentVariable(
        "Path", 
        [Environment]::GetEnvironmentVariable('Path', 
        [EnvironmentVariableTarget]::Machine) + ";$($Env:JAVA_HOME)\bin", 
        [EnvironmentVariableTarget]::Machine)
    \end{minted}

    Para probar que la instalación se haya hecho de forma correcta, cierren y abran 
    \textit{Powershell} y ejecuten el comando:

    \begin{minted}{powershell}
      java -version
    \end{minted}

    El input debiera ser algo como:

    \begin{minted}{text}
      openjdk version "14" 2020-03-17
      OpenJDK Runtime Environment (build 14+36-1461)
      OpenJDK 64-Bit Server VM (build 14+36-1461, mixed mode, sharing)
    \end{minted}
  %
%