\chapter{\textit{Java Development Kit (JDK)}}
  Para este curso utilizaremos \textit{Java} como el lenguaje predominante para ilustrar y
  evaluar los conceptos.

  La mayoría de las cosas que se verán en el semestre son agnósticas al lenguaje que se 
  utilice, y no se necesitan conocimientos previos en \textit{Java}.
  La segunda parte de este apunte se encargará de introducir el lenguaje, su modo de uso y
  algunas de las herramientas que éste provee.

  Se espera que el lector tenga un conocimiento básico de \textit{Python} y nociones 
  generales del lenguaje \textit{C} puesto que algunos de los ejemplos que se darán se 
  contrastarán con el comportamiento de estos respecto a \textit{Java}.

  \section{Instalación}
    Para instalar \textit{Java}, primero debe instalarse el compilador, la máquina virtual
    y la librería estándar.
    Todas estas herramientas vienen incluidas dentro del \textit{JDK}.

    Existen muchas maneras de instalar \textit{Java}, y a continuación se mostrarán 
    algunas.
    Las instrucciones siguientes son para instalar \textit{Java 13}, que es la versión más
    reciente al momento de escribir este apunte.
    Para el curso no es necesario que se utilice la última versión, pero es necesario que
    al menos usen \textit{Java 9} y recomendable que instalen al menos \textit{Java 11}.
    Si habían instalado anteriormente \textit{Java 8}, les recomentamos que 
    \textbf{desinstalen dicha versión} antes de proceder a instalar la más nueva porque 
    puede generar conflictos.

    \subsection{Opción 1: \textit{Open JDK} (Recomendado)}
      \subsubsection{Linux}
      %
    %
    \subsection{Windows}
      \subsubsection{Chocolatey (Recomendado)}
        Si no lo tienen instalado, el primer paso sería instalar el gestor de paquetes 
        \textit{Chocolatey}, para esto se debe abrir \textit{Powershell} como 
        administrador.

        \begin{minted}{powershell}
          [Net.ServicePointManager]::SecurityProtocol = [Net.SecurityProtocolType]::Tls12
          Set-ExecutionPolicy -ExecutionPolicy RemoteSigned -Force
          Invoke-WebRequest "https://chocolatey.org/install.ps1" -UseBasicParsing | Invoke-Expression
        \end{minted}

        Una vez que tengan \textit{Chocolatey} instalado, basta ejecutar:
      %
    %
  %
%