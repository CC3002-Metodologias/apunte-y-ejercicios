% !TEX root = D:\Users\Ignacio\Documentos\Escuela\CC3002 - Metodologías de Diseño y Programación\apunte-y-ejercicios\src\latex\Apunte.tex
\subsection{\textit{MacOS}}

  \subsubsection{Opción 1: \textit{Homebrew} (Recomendado)}
    Lo primero es instalar \textit{Homebrew} si no lo han hecho ya, para esto ejecuten:

    \begin{minted}{bash}
      /bin/bash -c "$(curl -fsSL https://raw.githubusercontent.com/Homebrew/install/master/install.sh)"
    \end{minted}

    Ya con \textit{Homebrew} instalado, basta correr el siguiente comando en una terminal:

    \begin{minted}{bash}
      brew install git
    \end{minted}
  %

  \subsection{Opción 2: \textit{Git for Mac}}
    Para esto deben descargar el instalador desde 
    \href{https://sourceforge.net/projects/git-osx-installer/}{aquí} y seguir las 
    instrucciones.

    Esta opción se sabe que puede tener problemas si se había instalado anteriormente otra
    versión de \textit{Git}.
  %
%