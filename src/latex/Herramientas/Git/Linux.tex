% !TEX root = D:\Users\Ignacio\Documentos\Escuela\CC3002 - Metodologías de Diseño y Programación\apunte-y-ejercicios\src\latex\Apunte.tex
\subsection{Linux}
  La forma más fácil de instalar \textit{Git} es utilizando el gestor de paquetes del
  sistema operativo que estén usando.
  A continuación se muestran los comandos necesarios para cada distribución de 
  \textit{Linux}:

  \subsubsection{
    \textit{Debian}\footnote{\textit{Ubuntu} es un sistema basado en \textit{Debian}}
  }
    \begin{minted}{bash}
      sudo apt install git-all 
    \end{minted}
  %

  \subsubsection{\textit{CentOS}}
    \begin{minted}{bash}
      sudo yum install git
    \end{minted}
  %

  \subsubsection{\textit{Fedora}}
    \begin{minted}{bash}
      sudo yum install git-core
    \end{minted}
  %

  \subsubsection{\textit{Arch Linux}\footnote{Por ejemplo \textit{Manjaro}.}}
    \begin{minted}{bash}
      sudo pacman -Sy git
    \end{minted}
  %

  \subsubsection{\textit{Gentoo}}
    \begin{minted}{bash}
      sudo emerge --ask --verbose dev-vcs/git
    \end{minted}
  %
%